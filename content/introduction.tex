\chapter{Introduction}
On the way to fully autonomously driving vehicles a multitude of challenges have to be overcome. A crucial part for autonomously driving vehicles is a planning system that typically encompasses different abstraction layers, such as mission, behavioral, and motion planning. While the mission planner provides a suitable route in a road network to arrive at the goal, the behavioral planner determines appropriate actions in traffic, such as changing lanes as well as stopping at intersections; the motion planner on the other hand operates on a lower level, avoiding obstacles, while progressing towards local goals \cite{Urmson.2008}.

An even more specialized problem is the navigation of the vehicle from a start pose to a goal pose in an environment that does not provide any specific structure; unstructured driving as opposed to structured driving (road following). Typical examples of such environments are parking lots or construction site. In these scenarios the vehicle needs to navigate safely around obstacles without any kind of path reference, such as lane markings, ideally using the optimal\footnote{with regard to a specified parameter, such as time, length, velocity, lateral acceleration, distance to obstacles, etc.}path between the start and the goal pose.

\section{Relevance of the Topic}
Autonomous driving vehicles might be one of the most publicly discussed and researched engineering topics at the moment of this writing. While the traditional car makers seem to pick up speed only very slowly, new entrants such as Alphabet (formerly Google) Comma.ai, drive.ai, Uber in the USA as well as OXBOTICA in the UK, and ZMP, Robot Taxi and nuTonomy in Asia are accelerating the advancement in self driving car technology at a rapid pace, releasing autonomous vehicles before the decade's end. A simple query with the search term \emph{autonomous driving} on Google Trends will underline this, revealing a huge increase in search volume over the past six years depicted in \Fref{fig:googleTrends}.

In a study titled ''Revolution in the Driver's Seat: The Road to Autonomous Vehicles'' the Boston Consulting Group defines Tesla's Autopilot released in October 2015 the first stage of several from partial to fully autonomous driving. Furthermore their estimates are, that fully autonomous driving vehicles will arrive by 2025. The most cited reasons by consumers in the USA to purchase an autonomous driving vehicle were an increase in safety, lower insurance premiums as well as the ability to be productive and multitask while the vehicle is driving. \cite{Mosquet.2015} 

For the society at large the most important benefit will be the reduction of accidents, as 90 percent of the time human error accounts for these. Other benefits of self-driving cars are the reduction of space consumption necessary for parking (self-driving cars do not need space to open the doors, which accounts to roughly 15\%) as reported by McKinsey \& Company. \cite{Bertoncello.2015}

\begin{figure}[h]
%\includegraphicsTex{MF.eps_tex}
\begin{tikzpicture}
	\begin{axis}[
		height=5cm,
		width=\textwidth,
		xlabel=Year, ylabel=Relative Frequency,
		xtick={40,197, 354, 510},
		xticklabels={Apr 2007, Apr 2010, Apr 2013, Apr 2016},
		]
		\addplot [black, mark=none]
		table [col sep=comma, x=week, y=freq] {figures/google_trends.dat};
%		\addplot [black, mark=-]
%		table[col sep=comma, x=week, y={create col/linear regression={y=freq}}] {figures/google_trends.dat};
	\end{axis}
\end{tikzpicture}
\caption{Google Trends for the query ``autonomous driving''}
\label{fig:googleTrends}
\end{figure}

\section{Context of the Thesis}
The Integrated Transport Research Lab (ITRL) at KTH is responding to the need for long-term multidisciplinary research cooperation to tackle the global environmental transport challenges by means of radically new and holistic technical solutions. ITRL's approach is that seamless transport services, infrastructure, novel vehicle concepts, business models and policies, all need to be tuned and optimized in chorus.

A key role for the research of new technical solutions at the ITRL is the Research Concept Vehicle (RCV), that serves as a test bed for a variety of research topics, that cover areas of the vehicle such as, design, control, perception, planning as well as systems integration.

As autonomous driving will be a part of the solution for global environmental transport challenges the aim with this and other theses is to equip the RCV step by step with the capabilities to become self-driving. Hence this thesis will be a part of a much larger project.

\section{Problem Description} \label{sec:problemDescription}
The problem that this thesis aims to solve can be summarized as follows.

\emph{Find a solution in real-time that transitions a nonholonomic vehicle collision free from a given start pose to a desired goal pose within an unstructured environment based on the input of a two dimensional obstacle map, or report the nonexistence of such a solution.}

\Fref{fig:problemDepiction} depicts the problem. The path transitions the vehicle from the start state $x_s$ safely to the goal state $x_g$, while adhering to the minimal turning radius $r$ of the vehicle and avoiding obstacles.

\begin{figure}[h]
\includegraphicsTex{problemDepiction.pdf_tex}
\caption{Depiction of a typical path planning problem to solve}
\label{fig:problemDepiction}
\end{figure}

\begin{figure}[h]
\includegraphicsTex{obstacleGrid.pdf_tex}
\caption[Obstacle grid]{Obstacle grid -- the input to the path planning algorithm}
\label{fig:obstacleGrid}
\end{figure}

\section{Scope and Aims}
The points listed below describe the key requirements regarding this thesis.

\begin{enumerate}
    \item Planning based on a local obstacle map
    \item Incorporating non-holonomic constraints
    \item Ensuring real-time capability
    \item Performing analysis and evaluation of results
    \item Integrating and driving autonomously (optional)
\end{enumerate}

The input to the path planning algorithm is a grid based binary obstacle map. The second point addresses the fact that the planner should be used for vehicles that cannot turn on the spot, e.g. cars, hence the produced paths must be continuous and need to be based on a model of the vehicle. In order to use the algorithm in a car, it needs to re-plan continuously and perform collision checking. For this purpose the actual implementation needs to be as efficient as possible, thus C++ is a necessity in order to provide the re-planning frequencies required. The development shall include a critical analysis and evaluation of the algorithm as well as its results. It is the aim to deploy the software on the RCV and demonstrate its capabilities in a real-world scenario, if time and other constraints will allow for it.

The scope of the thesis is to prototypically develop as well as implement a path planning algorithm, hence this focuses on a specific solution rather than a general and will not provide a comparison between different approaches as other works might do.

\section{Structure of the Thesis}
This thesis is split into two distinct parts. The first part of this thesis will present the theoretical foundation, in order to prepare the reader for the actual implementation of the algorithm. Chapter 2 shortly introduces the vehicle platform the path planning is being developed for. Chapter 3 deals with a brief introduction into path planning, where the term is dissected and popular approaches are touched on. Chapter 4 focuses on collision detection, as it is a vital part for most path planning approaches. And Chapter 5, the last chapter of the first part dissects popular graph search algorithms, that form the basis of this work.

The second part focuses on the method and its implementation in detail. Chapter 7 uses the theoretical framework to explain the implemented hybrid A* search in detail. Chapter 8 is primarily a collection of the results as well as analysis of the same. Chapter 9 Summarizes the entirety of the thesis, while stressing the achievements of this particular implementation. And finally chapter 10 will give the reader some suggestions regarding future work that can or should be conducted.