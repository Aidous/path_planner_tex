\chapter{Path Planning}
\label{chap:pathPlanning}
Fully autonomous robots need to be able to understand high-level commands, as it cannot be the goal to tell the robot how to to a certain thing, but rather what to do \cite{Latombe.1991}. While autonomous robots need to be able to reason, perceive and control themselves adequately, planning plays a key role. Planning consumes the robot's knowledge about the world to provide its actuators (controllers) with appropriate actions (references) to perform the task at hand. Machine learning may still be one of the largest area in artificial intelligence, but planning can be seen as a necessary complement to it, as in the future decisions need to be formed autonomously based on the learned \cite{LaValle.2006}. 

This section will present and discuss a variety of different planning techniques, many of which will not only be applicable to path planning, but are powerful tools for more general problems \cite{LaValle.2006}.

\section{Planning}
First, one should understand the meaning of the word \emph{planning}. Meriam Webster gives the following simple definition:``the act or process of making a plan to achieve or do something''. Given this general description it becomes obvious that one can mean very different things by saying the word planning, the following can all be seen as acts of planning.
\begin{enumerate}
	\item A person who intends to use public transportation to visit someone.
	\item A politician who signs a bill to persuade voters.
	\item A navigation system that calculates a route for a trip.
\end{enumerate}

Planning in this thesis is understood as the search for a set of actions $p$ that transitions a given start state to a desired goal state. When planning is conducted by computers one has to do it programmatically, the result is thus a \emph{planning algorithm} that usually returns the set of actions that transitions the start to the goal state. Planning under uncertainty is not considered, such that statement number 2 is not considered in this thesis.

\section{Path}
If planning returns the set of actions that are necessary to transition from a start to a goal state, then one can say that a path $\calP$ would be the entire set of actions, $p \in \calP$. More specifically a path in this thesis is the set of actions that move a vehicle from a current start state to a desired goal state through the Euclidean two dimensional plane.

While Latombe uses the term motion planning, Lavalle uses path planning, while their might some slight differences in the way they can be interpreted they have been used for the same things. %This thesis has adopted the term \emph{path planning}.

\section{Basic Problem}
The problem addressed in this thesis shall be further specified and a general formulation be established. The robot is the only moving object in the world. The dynamics of the robot are not considered, hence removing any time dependencies. As collisions shall not occur they will not be modeled either.

Based on these simplifications the basic problem can be described as follows:
\begin{itemize}
\item $\calA \subset \calW$, the robot, is a single moving rigid object in world $\calW$ represented in the Euclidean space as $\R^2$
\item $\calO \subset \calW$, the obstacles, are stationary rigid objects in $\calW$
\item geometry, position and orientation of $\calA$ and $\calO$ are known a priori
\item \textbf{Problem:} given a start and goal pose of $\calA \subset \calW$, plan a path $\calP \subset \calW$ denoting the set of poses so that $\calA(p) \cap \calO = \emptyset$ for any pose  $p \in \calP$ along the path from start to goal, terminate and report $\calP$ or $\emptyset$, if a path has been found or no such path exists
\end{itemize}

\cite{Latombe.1991,LaValle.2006}

\section{Configuration Space}
In the literature of motion planning the concept of the configuration space has been well established, as it facilitates the formulation of path planning problems, presenting various concepts with an underlying scheme yielding greater expressive power \cite{LaValle.2006,Latombe.1991}. Configuration spaces have been extensively used by Lozano-P\'{e}rez in the 80s for the description of spatial planning problems, e.g. the search for an appropriate space to place an object or the search for a path of an object in an unstructured environment with obstacles \cite{Latombe.1991}. The search for the appropriated placement or the path an object should take, is a problem often encountered in design and manufacturing, where compactness needs to be achieved but maintainability shall not be sacrificed. The advantage of the configuration space representation is that it reduces the problem from a rigid body to a point and thus eases the search \cite{LozanoPerez.1983}.

Given the robot or agent $\calA \subset \calW$, where $\calW$ is the world as the two dimensional Euclidean space $\R^2$ with a fixed Cartesian coordinate frame $\calF_\calW$, each possible configuration of $\calA$ can be described by $q$ of the form $(x,y,\theta)$ denoting a position along the $x$- and $y$-axis as well as an orientation in $\calF_\calW$ respectively. All possible configurations $q$ form the configuration space $\calC \subset \calW$. Attached to the robot $\calA$ is the frame $\calF_\calA$, so that $\calF_\calW(q) = \calF_\calA$, this allows the description of parts of the robot relative to $\calA$ and not $\calW$. The configuration space of the robot can be further divided in subsets $\calC_{obs}$ and $\calC_{free}$. The configuration space $\calC_{obs} \subset \calC$ describes the set of all $q$ where $\calA(q) \cup \calO \neq \emptyset$ and hence the robot is in collision. On contrast $\calC_{free} \subset \calC$ is the set of all safe configurations $q$, $\calA(q) \cup \calO = \emptyset$. \cite{LaValle.2006,Latombe.1991}

Since the configuration space only considers the static case, the regions defined are not fully descriptive, once dynamics are introduced. In order to incorporate dynamics, analog to $C$ the state space $X$ can be introduced, with $X = \calC$. In addition to $\calC$, $X$ allows the description of the dynamics of the robot so that with $f: x \to q$ a given state $x$ with $(x,y,\theta,\dot{x},\dot{y})$ can be mapped to $q$ as $(x,y,\theta)$. Describing this will add another subset $X_{ric} \to \calC_{free}$ to $X$, the region of inevitable collision. This region describes all states that will lead to a collision due to robot's dynamics as $\calC_{X_{ric}} \subset \calC_{free}$. \cite{LaValle.2006,Latombe.1991}

\begin{figure}[h]
    \includegraphicsTex{MF.eps_tex}
    \caption{Configuration space of a robot}
    \label{fig:configurationSpace}
\end{figure}

\section{Popular Approaches}
As with most problems, there are a variety of different methods for path planning based on the requirements of the problem to be solved. The aim of the following is not to compare methods for path planning. Each has advantages and disadvantages, as discussed in many other places, but it should rather give the reader a quick overview over different type of solutions for the same problem. %In the area of sampling-based methods Probabilistic Roadmaps (PRMs), Rapidly-exploring Random Trees (RRTs) as well as State Lattice planners are among the most popular to solve problems in higher dimensional state spaces.\cite{LaValle.2006,Latombe.1991}

\subsection{RRT}
Rapidly-exploring random trees were first introduced in the late 90s by Lavalle and Kuffner. The goal was to create an algorithm that could be applied to general nonholonomic as well as kinodynamic planning problems, which other randomized approaches such as the randomized potential field and probabilistic road map struggled with \cite{Lavalle.1998}. The basic structure is exemplified in the algorithm below.

A rapidly-exploring random tree (RRT) grows by repeatedly selecting a random state in a bounded region algorithm \algref{alg:RRT}{alg:RRTrandomState}. Starting from this random state the nearest neighbor is selected in line \ref{alg:RRTnearestNeighbor}. A new state is created in line \ref{alg:RRTnewState} by using an appropriate control action, line \ref{alg:RRTcontrolAction}. During this phase it needs to be ensured that the resulting edge will be collision free. Finally in line \ref{alg:RRTaddVertex} and \ref{alg:RRTaddEdge} the tree is being grown by adding the new state and the connecting edge. Using this approach the tree explorers the state space in a rapid and uniform manner as it probabilistically grows away from its root node \ref{fig:RRT} expanding towards larger Voronoi regions due to its randomization. Inherent to their nature RRTs are non guided and hence non optimal. \cite{Lavalle.1999}

An RRT can however be biased by changing the probability distribution. Sampling in the goal region, allows for much faster convergence. Another approach is to include a measure that takes the path costs into account. \cite{Lavalle.1999}

Urmson presents a heuristically-guided RRT where the probability distribution takes the size of the Voronoi region into account as well as the quality of the path so far \cite{Urmson.2003}.

\begin{algorithm}
    \caption{Rapidly-exploring Random Tree}\label{alg:RRT}
    \begin{algorithmic}[1]
        \Require $x_s \cap x_g \in X_{free}$
        \State $T.init(x_s)$
        \While{$x_{new} \neq x_g$}
            \State $x_{rnd} \gets randomState()$ \label{alg:RRTrandomState}
            \State $x_{ngh} \gets nearestNeighbor(x_{rnd})$ \label{alg:RRTnearestNeighbor}
            \State $u \gets controlAction(x_{rnd}, x_{ngh})$ \label{alg:RRTcontrolAction}
            \State $x_{new} \gets newState(x_{rnd}, x_{ngh}, u)$ \label{alg:RRTnewState}
            \State $T.addVertex(x_{new})$ \label{alg:RRTaddVertex}
            \State $T.addEdge(x_{ngh}, x_{new}, u)$ \label{alg:RRTaddEdge}
        \EndWhile
        \State \textbf{return} $T$
    \end{algorithmic}
\end{algorithm}

\begin{figure}[h]
    \includegraphicsTex{MF.eps_tex}
    \caption{Rapidly-exploring random tree biased towards unexplored areas}
    \label{fig:RRT}
\end{figure}

\subsection{Potential Fields}
Potential Fields were originally used as an online collision avoidance measure for static as well as moving obstacles using the artificial potential field concept to make the collision avoidance part of lower level of control, decreasing response time. The idea behind the artificial potential field approach can be summarized, with the robot moving in a field of forces. The goal position is an attractive, while an obstacle creates a repulsive pole \cite{Khatib.1986}. The shape of the artificial potential field $U$ represents the layout of the configuration space. Path planning is conducted in a sequential manner. In each step the artificial force $\vec{F}(q) = -\vec{\triangledown}U(q)$ acting on the robot's current configuration will create a path increment towards this direction. \cite{Latombe.1991}

As obstacle avoidance was the original goal, potential fields come with the risk of not being able to reach the goal as the robot might get stuck in local minima. To work around this inherent characteristic one can try to formulate the potential function without local minima or incorporate techniques that allow the escape from the same, such as RRT \cite{LaValle.2006}. \cite{Latombe.1991}

\begin{figure}[h]
    \includegraphicsTex{MF.eps_tex}
    \caption{Potential Fields}
    \label{fig:potentialFields}
\end{figure}

\subsection{Approximate Cell Decomposition}
Path planning with approximate cell decomposition can be traced back to Brooks and Lozano-Peréz in the mid 80s. The basic idea is to divide the configuration space into rectangles with edges parallel to the axes of the space. The resulting cells will be either free, occupied or mixed, depending on the configuration space of the obstacles intersecting with the respective rectangle or not, see \Fref{fig:cellDecomposition}. The search for a path is conducted by finding a set of connected and free cells that include the start as well as the goal configuration.\cite{Brooks.1985} Applicable graph search algorithms are explained in more detail in \fref{chap:graphSearch}.

This approach does not represent the free space exactly, hence a conservative approach must be taken in order to avoid faulty planning. Decomposing the cells in this way, although not exact, in general allows for planning methods that are easier to implement in comparison to exact cell decomposition. \cite{Latombe.1991}

\begin{figure}[h]
    \includegraphicsTex{MF.eps_tex}
    \caption{Approximate Cell Decomposition}
    \label{fig:cellDecomposition}
\end{figure}

\section{Differential/Kinematic Constraints}
This section focuses on the differential constraints of a car-like robot. Most of the time these differential constraints are inherent to the kinematics and dynamics of the robot itself. These constraints need to be taken into account at some point, ideally during the actual path planning process ensuring the path matching the robot's constraints. In case that considering the constraints during the planning process is infeasible one could also unload this task on the controller, which given the constraints will not be an easy task either. \cite{LaValle.2006}

Even though a car can reach any position and orientation in the Euclidean plane, with a configuration as $q = (x,y,\theta)$, its configuration space is thus $\calC = \R^2 \times \S^1$, it cannot translate or rotate freely. A car can move forward as well as backward, but it cannot move sideways. This means that there are less possible actions than degrees of freedom, systems such as this are called underactuated, \cite{LaValle.2006}.

A common case for drivers of cars is to parallel park, in order to achieve a configuration $q_1$ that is parallel to $q_0$ it is necessary for a car to at least rotate and translate the same goes for a third configuration $q_2$ that has the same position but different orientation than $q_0$. \cite{Latombe.1991}

\begin{figure}[h]
    \includegraphicsTex{MF.eps_tex}
    \caption{Kinematic Constraints of a car-like robot}
    \label{fig:kinematicConstraints}
\end{figure}

\subsection{Reeds-Shepp Car}

\subsection{Dubins Car}