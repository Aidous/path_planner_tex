\section{Path Smoothing}
As the paths produced by the hybrid A* algorithm are drivable, but often are made up of unnecessary steering actions it is beneficial to post process the result with a smoother that attains a higher degree of comfort and safety \cite{Dolgov.2008,Dolgov.2010}. For this purpose a gradient descent smoother can be used that aims to minimize $P$ consisting of the following four terms with respect to the path.

\begin{equation}
P = P_{obs} + P_{cur} + P_{smo} + P_{vor}
\end{equation}

Each of the terms in this cost function has a special purpose that shall be explained in more detail.

\subsection{Obstacle Term}

This term penalizes collisions with obstacles. For all vertices $\bldx_i$ where $|\bldx_i - \bldo_i| \leq d_{obs}^{\lor}$ the cost $P_{vor}$ is defined. It based on the the distance to the next obstacle.

%\begin{figure}[h]
%    \includegraphicsTex{MF.eps_tex}
%    \caption{Impact of the obstacle term}
%    \label{fig:obstacleTerm}
%\end{figure}

\begin{equation}
P_{obs} = w_{obs} \displaystyle\sum_{i=1}^{N} \sigma_{obs}(|\bldx_i-\bldo_i|-d_{obs}^{\lor})
\end{equation}

Where $\bldx_i$ is the $x,y$-position of a vertex on the path, $\bldo_i$ the location of the closest obstacle to $\bldx_i$. $d_{obs}^{\lor}$ acts as a threshold for the the maximum distance obstacles can affect the cost of the path. In order to penalize heavier when getting close to obstacles $\sigma_{obs}$ is a quadratic penalty function. The obstacle weight $w_{obs}$ is used to influence the impact on the change of the path.

\paragraph{Gradient}

\begin{equation}
\frac{\partial \sigma_{obs}}{\partial \bldx_i} = \frac{2(|\bldx_i-\bldo_i|-d_{obs}^{\lor})\bldx_i-\bldo_i}{|\bldx_i-\bldo_i|}
\end{equation}

\subsection{Curvature Term}
In order to ensure driveability the curvature term upper-bounds the instantaneous curvature of the path at every vertex. It is defined for $\frac{\Delta\phi_i}{|\Delta\bldx_i|} > \kappa_{max}$

%\begin{figure}[h]
%    \includegraphicsTex{MF.eps_tex}
%    \caption{Impact of the curvature term}
%    \label{fig:curvatureTerm}
%\end{figure}

\begin{equation}
P_{cur} = w_{cur} \displaystyle\sum_{i=1}^{N-1} \sigma_{cur}\left(\frac{\Delta\phi_i}{|\Delta\bldx_i|} - \kappa_{max}\right)
\end{equation}

The displacement vector at the vertex $\bldx_i$ is defined as $\Delta\bldx_i= \bldx_i - \bldx_{i-1}$. The change in tangential angle at a vertex can be expressed by $\Delta\phi_i$ = $\cos^{-1}\frac{\bldx_{i}\cdot\bldx_{i+1}}{|\bldx_{i+1}||\bldx_{i+1}| }$.  The maximum allowable curvature is denoted by $\kappa_{max}$. Deviations from the maximum allowable curvature are penalized with a quadratic penalty function $\sigma_{cur}$. The curvature weight $w_{cur}$  controls the impact on the change of the path.

\paragraph{Gradients}

\begin{equation}
\frac{\partial \kappa_i}{\partial \bldx_i} = \frac{1}{|\Delta\bldx_i|}\frac{\partial\Delta\phi_i}{\partial\cos\Delta\phi_i}\frac{\partial\cos\Delta\phi_i}{\partial\bldx_i}-\frac{\Delta\phi_i}{{\Delta\bldx_i}^2}\frac{\partial\Delta\bldx_i}{\partial\bldx_i}
\end{equation}

\begin{equation}
\frac{\partial \kappa_i}{\partial \bldx_{i-1}} = \frac{1}{|\Delta\bldx_i|}\frac{\partial\Delta\phi_i}{\partial\cos\Delta\phi_i}\frac{\partial\cos\Delta\phi_i}{\partial\bldx_{i-1}}-\frac{\Delta\phi_i}{{\Delta\bldx_i}^2}\frac{\partial\Delta\bldx_i}{\partial\bldx_{i-1}}
\end{equation}

\begin{equation}
\frac{\partial \kappa_i}{\partial \bldx_{i+1}} = \frac{1}{|\Delta\bldx_i|}\frac{\partial\Delta\phi_i}{\partial\cos\Delta\phi_i}\frac{\partial\cos\Delta\phi_i}{\partial\bldx_{i+1}}
\end{equation}

\subsection{Smoothness Term}
The smoothness term evaluates the displacement vectors between vertices. The result is that it assigns cost to vertices that are unevenly spaced as well as change direction. $w_{smo}$ denotes the smoothness weight and hence the impact of the term on the change of the path.

%\begin{figure}[h]
%    \includegraphicsTex{MF.eps_tex}
%    \caption{Impact of the smoothness term}
%    \label{fig:smoothnessTerm}
%\end{figure}

\begin{equation}
P_{smo} = w_{smo} \displaystyle\sum_{i=1}^{N-1} (\Delta\bldx_{i+1} - \Delta\bldx_i)^2
\end{equation}

\subsection{Voronoi Term}
This term guides the path away from obstacles. For $d_{obs} \leq d_{vor}^{\lor}$ the cost $P_{vor}$ is defined. It is based on the position of the node in the Voronoi field.

%\begin{figure}[h]
%    \includegraphicsTex{MF.eps_tex}
%    \caption{Impact of the Voronoi term}
%    \label{fig:VoronoiTerm}
%\end{figure}


\begin{equation}
P_{vor} = w_{vor} \displaystyle\sum_{i=1}^{N} \left(\frac{\alpha}{\alpha + d_{obs}(x,y)}\right)\left(\frac{d_{vor}(x,y)}{d_{obs} + d_{vor}(x,y)}\right)\left(\frac{(d_{obs}(x,y) - d_{vor}^{\lor})^2}{(d_{vor}^{\lor})^2}\right)
\end{equation}

The positive distance to the nearest obstacle is denoted by $d_{obs}$, $d_{edg}$ is the positive distance to the nearest edge of the GVD. $d_{vor}^{\lor}$ represents the maximum distance obstacles affect the Voronoi potential. $\alpha > 0$ controls the falloff rate of the field and $w_{vor}$, the Voronoi weight, influences the impact on the path.

\paragraph{Gradients}

\begin{equation}
\frac{\partial d_{obs}}{\partial \bldx_i} = \frac{\bldx_i-\bldo_i}{|\bldx_i-\bldo_i|}
\end{equation}

\begin{equation}
\frac{\partial d_{edg}}{\partial \bldx_i} =\frac{\bldx_i-\blde_i}{|\bldx_i-\blde_i|}
\end{equation}

\begin{equation}
\frac{\partial\rho_{vor}}{\partial d_{obs}} = \frac{\alpha d_{edg}\left(d_{obs}-d_{vor}^{\lor}\right)\left(\left(d_{edg}+2d_{vor}^{\lor}+\alpha\right) d_{obs}+\left(d_{vor}^{\lor}+2\alpha\right)d_{edg}+\alpha d_{vor}^{\lor}\right)}{{d_{vor}^{\lor}}^2\left(d_{obs}+\alpha\right)^2\left(d_{obs}+d_{edg}\right)^2}
\end{equation}

\begin{equation}
\frac{\partial\rho_{vor}}{\partial d_{edg}} =  \frac{\alpha d_{obs}\left(d_{obs}-d_{vor}^{\lor}\right)^2}{{d_{vor}^{\lor}}^2\left(d_{obs}+\alpha\right)\left(d_{edg}+d_{obs}\right)^2}
\end{equation}

\subsection{Gradient Descent}
The gradient descent method is an optimization algorithm that uses the gradient of the function in order to find a local minimum. Gradient descent progresses stepwise with a step size proportional to the negative gradient of the function, $\Delta x = - \nabla f(x)$. While usual implementations use the absolute value of the gradient as a stopping criterion, a fixed number of iterations was chosen to ensure run-time consistency. \cite{Boyd.2004}

\begin{algorithm}
    \caption{Gradient Descent}\label{alg:gradientDescent}
    \begin{algorithmic}[1]
    \State $iterations \gets 1000$
    \State $i \gets 0$
        \While{$i < iterations$}
        	\ForAll {$\bldx \in \calP$}
        		\State $cor \gets  (0,0)$
        		\State $cor \gets  cor - obstacleTerm(\bldx_{i})$
        		\State $cor \gets  cor - smoothnessTerm(\bldx_{i-1},\bldx_{i},\bldx_{i+1})$
        		\State $cor \gets  cor - curvatureTerm(\bldx_{i-1},\bldx_{i},\bldx_{i+1})$
        		\State $cor \gets  cor - voronoiTerm(\bldx_{i})$
        		\State $\bldx_{i} \gets \bldx_{i} + cor$
        	\EndFor
        \State $i \gets i + 1$
        \EndWhile
        \State \textbf{return} null
    \end{algorithmic}
\end{algorithm}